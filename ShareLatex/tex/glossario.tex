
\newword{Framework}{É uma abstração que une código comum entre vários projetos de software que fornecem uma funcionalidade genérica. Frameworks são concebidos com a intenção de facilitar o desenvolvimento de software, permitindo que designers e desenvolvedores de passar mais tempo na determinação dos requisitos de software do que com detalhes de baixo nível do sistema. Software Framework compreende um conjunto de classes implementadas em uma linguagem de programação específica utilizada para apoiar o desenvolvimento de software.}

\newword{Framework Conceitual}{A estrutura teórica de pressupostos, princípios e regras que une as idéias que compreende um conceito amplo. Framework Conceitual, não é um software executável, mas um modelo de dados para um domínio.}

\newword{Artifact}{É o produto de uma ou mais atividades dentro do contexto de um desenvolvimento de software. Assim, cada fase / iteração do desenvolvimento irá resultar em um documento (por exemplo fluxograma, DER, fluxo de trabalho, diagramas UML ou qualquer outro documento) que servirá como uma fonte de informação.}


\newword{Tecnologia Assistiva}{Ele refere-se a produtos, recursos, metodologias, estratégias, práticas e serviços que visam promover a funcionalidade de pessoas com deficiência, deficiência ou mobilidade reduzida, por sua autonomia, independência, qualidade de vida e inclusão social.}

\newword{Realidade Virtual}{É uma tecnologia de interface avançada entre um usuário e um sistema de computador. O objetivo desta tecnologia é recriar ao máximo a sensação de realidade para um indivíduo. Portanto, essa interação acontece em tempo real, utilizando técnicas e equipamentos computacionais para auxiliar na expansão do sentimento de presença do usuário.}

\newword{Webservices}{É uma solução usada em sistemas de integração e comunicação entre diferentes aplicações. Com esta tecnologia é possível que novas aplicações possam interagir com as que já existam e que sistemas desenvolvidos em diferentes plataformas sejam compatíveis.}