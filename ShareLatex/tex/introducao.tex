\section{Contextualização}
\label{Contextualization}

A informação é matéria-prima para as organizações e as auxilia a sobreviver no mercado competitivo. A informação permite a interação entre diferentes departamentos e, também, possibilita aos gestores obter uma análise mais ampla da empresa. Cada indivíduo, no âmbito da organização, elabora um entendimento próprio sobre determinada informação, conforme afirma \cite{Choo2003} “a informação é fabricada por indivíduos a partir de sua experiência passada e de acordo com as exigências de determinada situação na qual a informação deve ser usada”.

As organizações estão começando a perceber a importância da informação. Atualmente, as pessoas em geral são expostas a uma grande quantidade de informação, sendo pressionadas a estar sempre bem informadas. Nesse contexto, torna-se necessário saber qual informação é relevante para determinada ação, seja simples ou complexa. Trata-se de uma questão de “inteligência”, ou seja, da habilidade para transformar a imensa massa de dados operacionais, que correm nas veias da empresa diariamente, em informações consistentes que agreguem valor ao negócio \cite{Teixeira2000}.

Dada a complexidade desses dados e como eles estão espalhados, faz-se necessário o uso de ferramentas que possam compreende-los automaticamente e, assim, apoiar tomadas de decisão mais rápidas e eficazes. Para compreender o significado em conjuntos de dados, programas necessitam associar informações semânticas a eles. Para isso, ontologias são usadas.

 Segundo Gruber (1995), uma ontologia é ”uma especificação explícita de uma conceitualização que representa o entendimento comum e a terminologia relevante de um domínio”. Ela é um sistema de organização e representação do conhecimento. Nesse contexto, Noy et al. (2001) relatam que as ontologias surgiram para compartilhar, organizar e especificar conhecimentos de um determinado domínio.
 
 Especialistas em um domínio de conhecimento, domain experts, são as pessoas indicadas para criar ontologias para esse domínio. Contudo, na maioria das vezes, esses especialistas não têm o domínio de ferramentas e linguagens para produzir ontologias que possam ser lidas por computadores.


\section{Motivação}

O tamanho e complexidade de ontologias cresce constantemente e as diversas origens dos usuários e áreas de aplicações multiplicam-se. Fornecer representações visuais e técnicas de interação intuitiva aos usuários pode ajudar significativamente a exploração e compreensão dos domínios representados por ontologias.

A Visualização/Edição de Ontologia não é um tema novo e um certo número de abordagens se tornaram disponíveis nos últimos anos. Algumas já estão bem estabelecidas, particularmente no campo da modelagem de ontologias. Em outras áreas da engenharia de ontologias, como alinhamento de ontologias e depuração, embora várias ferramentas tenham sido recentemente desenvolvidas, poucas fornecem uma interface gráfica do usuário, para não mencionar ajudas à navegação ou técnicas de visualização e interação abrangentes.

Na presença de uma enorme rede de recursos interligados na web, chamada de Linked Open Data (LOD), um dos desafios enfrentados pela comunidade é a visualização de conjuntos de dados multidimensionais para proporcionar uma visão geral eficiente. Com o foco passando de uma Web de documentos para uma Web de dados, mudanças nos paradigmas de interação estão em demanda também. Essas novas abordagens também precisam levar em consideração os desafios tecnológicos e oportunidades dadas por novos contextos de interação, por exemplo, toque e interação gestual.

Não há uma solução única, mas diferentes casos de uso que exigem diferentes técnicas de visualização e interação. Em última análise, proporcionando melhores interfaces de usuário, representações visuais e técnicas de interação irá promover o envolvimento dos usuários e, provavelmente, conduzem a resultados de maior qualidade em diferentes aplicações que empregam ontologias e assim, proliferar o consumo de dados vinculados.

Para que esta ferramenta possa ser desenvolvida, é importante identificar os requisitos que atendam às necessidades dos usuários, especialistas e das possíveis modelagens da Ontologia. Os estudos discutidos em \autoref {trabalhosrelacionados} não identificam quais requisitos foram utilizados para desenvolver ferramentas de sistema existentes. Assim, um quadro conceitual precisa ser desenvolvido com o objetivo de ajudar os desenvolvedores a criar ferramentas computacionais, a fim de apoiar os Domains Experts.

\section{Objetivos}

Com a finalidade de fornecer uma ferramenta de edição/visualização gráfica de Ontologias, que possa ser usada por domain experts, propõem-se, neste trabalho, o desenvolvimento de um sistema computacional, baseado na web, para a edição de ontologias.

A visualização gráfica, através de grafos e árvores, e a possibilidade do uso de comandos, numa linguagem próxima à usada por domain experts, vai permitir a criação/edição de ontologias por usuários fora da área da computação, com o mínimo de treinamento.

Este projeto de mestrado tem como objetivo propor e avaliar um editor de Ontologia Visual como ferramenta de apoio para construção de sistemas inteligentes, permitindo assim, que a partir de elementos visuais seje criado e/ou utilizada a Ontologia. O objetivo principal deste projeto pode ser dividido nos seguintes objetivos específicos:

\begin{itemize}
\item Permitir que o editor utilize Ontologias já criadas;
\item Permitir que se acrescente outras propriedades a Ontologias já criadas;
\item Criação de novas Ontologias para serem utilizadas;
\item Exportar os dados em formato compativel com \sigla{W3C}{World Wide Web Consortium} (\sigla{RDF}{Resource Description Framework} e/ou \sigla{OWL}{Web Ontology Language});
\end{itemize}

Para se estabelecer um quadro conceitual para o desenvolvimento do editor de Ontologia Visual, os requisitos devem ser identificados através da realização de uma revisão da literatura na área, e com entrevistas à especialistas. Posteriormente, esses requisitos recolhidos devem ser analisados e utilizados para construir os artefatos que compõem o quadro conceitual. Esses artefatos podem ser fluxos de trabalho, fluxogramas, diagramas de \sigla{UML}{Unified Modeling Language} ou \sigla{ERD}{Entity Relationship Diagram}. Para finalizar, é de suma importancia a construição de um protótipo para avaliar o quadro conceitual. Seguindo estes passos, é esperado que a todo o trabalho possa contribuir para o desenvolvimento de editor e que o mesmo possa estar apto a ser utilizado como elemento alternativo a escrita de Ontologias.

\section{Estrutura Proposta}

Esta tese de qualificação é organizada da seguinte forma: No capítulo 2 é apresentado os principais conceitos utilizados no desenvolvimento deste trabalho. No capítulo 3, os resultados do mapeamento sistemático são apresentados e no capítulo 4 é apresentado o trabalho proposto.